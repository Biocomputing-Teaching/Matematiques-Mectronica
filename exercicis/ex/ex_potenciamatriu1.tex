\Exercise Calcula $\begin{pmatrix}1&1\\3&-1\end{pmatrix}^6$ usant el concepte de valors i vectors propis de la matriu.

\Answer
Sabem que si una matriu $A$ és diagonalitzable, podem expressar-la com
\[D=P^{-1}AP\]
on $D$ és una matriu diagonal amb els valors propis de la matriu $A$ i $P$ és una matriu quines columnes representen els vectors propis associats a cada valor propi.

Si usem aquesta expressió, fer l'operació és senzilla. Efectivament:
\[\left(PDP^{-1}\right)^6=A^6\]
o, el que és el mateix:
\begin{eqnarray}
  PDP^{-1}PDP^{-1} PDP^{-1} PDP^{-1}PDP^{-1}PDP^{-1}&=&A^6\nonumber\\
  PD^6P^{-1}&=&A^6\label{eq_A6}
\end{eqnarray}


Anem, doncs, a cercar els valors i vectors propis de la matriu. Aquests seran els valors que compleixin:

\[
  \begin{pmatrix}1&1\\3&-1\end{pmatrix}
  \begin{pmatrix}x\\y\end{pmatrix}=\lambda\begin{pmatrix}x\\y\end{pmatrix}\]

  Perquè això tingui solució diferent de la trivial ($(x,y)=(0,0)$) cal que el determinant secular sigui igual a zero:
  \[
  \begin{vmatrix}1-\lambda&1\\3&-1-\lambda\end{vmatrix}=0  
  \]
  Per tant, cal solucionar el polinomi característic de la matriu:
  \[(1-\lambda)(-1-\lambda)-3=0\]
  d'on $\lambda=\pm 2$.


Un cop tenim els valors propis, trobem els vectors propis que hi estan vinculats:

\begin{description}
  \item[$\boxed{\lambda_1=2}$] 
  \[
    \begin{pmatrix}1&1\\3&-1\end{pmatrix}
  \begin{pmatrix}x\\y\end{pmatrix}=2\begin{pmatrix}x\\y\end{pmatrix}
  \]
  Solucionem el sistema:
  \[
    \systeme*{x+y=2x,3x-y=2y} \Rightarrow \systeme*{-x+y=0,x-y=0} \Rightarrow x=y
  \]
  Per tant, un vector propi\footnote{De fet, el conjunt de vectors propis associats a un determinat valor propi forma un subespai vectorial.} associat a $\lambda_1=2$ és $\uvec{v}_1=\begin{pmatrix}1\\1\end{pmatrix}$. 
  \item[$\boxed{\lambda_1=-2}$] 
  \[
    \begin{pmatrix}1&1\\3&-1\end{pmatrix}
  \begin{pmatrix}x\\y\end{pmatrix}=-2\begin{pmatrix}x\\y\end{pmatrix}
  \]
  Solucionem el sistema:
  \[
    \systeme*{x+y=-2x,3x-y=-2y} \Rightarrow 3x+y=0
  \]
  Per tant, un vector propi associat a $\lambda_2=2$ és $\uvec{v}_2=\begin{pmatrix}1\\-3\end{pmatrix}$. 
\end{description}

Construïm ara l'expressió de l'Eq. \ref{eq_A6}:

\begin{eqnarray*}
A^6 &=&\begin{pmatrix}1&1\\1&-3\end{pmatrix}\begin{pmatrix}2&0\\0&-2\end{pmatrix}^6\begin{pmatrix}1&1\\1&-3\end{pmatrix}^{-1}\\
&=&\begin{pmatrix}1&1\\1&-3\end{pmatrix}\begin{pmatrix}64&0\\0&64\end{pmatrix}\begin{pmatrix}\frac{3}{4}&\frac{1}{4}\\\frac{1}{4}&-\frac{1}{4}\end{pmatrix}\\
&=&\begin{pmatrix}64&0\\0&64\end{pmatrix}
\end{eqnarray*}

Sabries dir d'on prové el fet que, en aquest cas particular, la potència de la matriu sigui idèntica a la potència de la seva matriu semblant diagonal?




\blacksquare
