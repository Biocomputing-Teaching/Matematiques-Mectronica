\Exercise Trobeu les equacions paramètriques i cartesianes de les rectes següents:
\begin{enumerate}
  \item La recta que passa pels punts $(1,0,1)$ i $(1,3,-2)$
  \item La recta que passa pel punt $(-2,0,3)$ i és paral·lela al vector $\uvec{v}=6\uvec{i}+3\uvec{j}$
  \item La recta que passa pel punt $(-3,5,4)$ i és paral·lela a la recta $\frac{x-1}{3}=\frac{y+1}{-2}=z-3$
\end{enumerate}

\Answer L'equació paramètrica que defineix un punt qualsevol de la recta a $\mathbf{R}^3$ és:
\[
\begin{pmatrix}x\\y\\z\end{pmatrix}= \begin{pmatrix}x_0\\y_0\\z_0\end{pmatrix}+\lambda\begin{pmatrix}v_x\\v_y\\v_z\end{pmatrix}
\]
on $P=(x_0,y_0,z_0)$ és un punt qualsevol de la recta i $\uvec{v}=(v_x,v_y,v_z)$ el seu vector director. en tots tres casos ens donen un punt de la recta i ens donen informació que ens en permet calcular el vector deirector.

\begin{enumerate}
  \item La recta que passa pels punts $(1,0,1)$ i $(1,3,-2)$

  Aquí $P=(1,0,1)$ i $\uvec{v}=(1,3,-2)-(1,0,1)=(0,3,-3)$. Per tant, l'equació paramètrica queda:
  \[
  \begin{pmatrix}x\\y\\z\end{pmatrix}= \begin{pmatrix}1\\0\\1\end{pmatrix}+\lambda\begin{pmatrix}0\\3\\-3\end{pmatrix}
  \]

  Pel que fa a l'equació cartesiana o general, la podem construir a partir de la contínua, que no és més que igualar el paràmere $\lambda$ provinent de cada coordenada a l'equació anterior. Així, l'eqació contínua seria (amb un petit abús denotació permetent-nos posar un $0$ al denominador):
  \[
  \frac{x-1}{0}=\frac{y}{3}=\frac{z-1}{-3}
  \]
  Agafant els dos primers termes de l'equació obtenim la primera de les equacions següents i agafant el segon i tercer terme obtenim la segona equació:
  \[
  \begin{cases}3x-3=0\\-3y=3z-3\end{cases}
  \]
  Simplificant:
  \[
  \begin{cases}x-1=0\\y+z-1=0\end{cases}
  \]
  Notar que l'equació d'una recta a $\mathbf{R}^3$ es construeix amb les equacions de dos plans que es tallen.

  \item La recta que passa pel punt $(-2,0,3)$ i és paral·lela al vector $\uvec{v}=6\uvec{i}+3\uvec{j}$

    En aquest cas ens donen directament $\uvec{v}=(6,3,0)$. Podem fer la mateixa operació que abans:
    \[
    \frac{x-2}{6}=\frac{y}{3}=\frac{z-3}{0}
    \]
    d'on
    \[
    \begin{cases}3x-6=6y\\0=3z-9\end{cases}
    \]
    Simplificant:
    \[
    \begin{cases}x-2y-2=0\\z-3=0\end{cases}
    \]

    \item La recta que passa pel punt $(-3,5,4)$ i és paral·lela a la recta $\frac{x-1}{3}=\frac{y+1}{-2}=z-3$

 Idènticament, ens diuen que $\uvec{v}=(4,1,-3)$. Fem la contínua i després la general o cartesiana:
 \[
 \frac{x+3}{3}=\frac{y-5}{-2}=\frac{z-4}{1}
 \]
 d'on
 \[
 \begin{cases}-2x-6=3y-15\\y-5=-2z+8\end{cases}
 \]
 Simplificant:
 \[
 \begin{cases}2x+3y-9=0\\y+2z-13=0\end{cases}
 \]
 \blacksquare


\end{enumerate}
