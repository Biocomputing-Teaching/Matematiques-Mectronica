\Exercise Trobeu les equacions cartesianes dels plans següents:
\begin{enumerate}
  \item El pla que passa pels punts $(0,0,0)$, $(1,2,3)$ i $(-2,3,3)$.
  \item El pla que passa pel punt $(2,1,2)$ i té per vector normal $\uvec{n}=2\uvec{i}+3\uvec{j}-\uvec{k}$.
  \item El pla que passa pel punt $(3,2,2)$ i és perpendicular a la recta $\frac{x-1}{4}=y+2=\frac{z+3}{-3}$
  \item El pla que conté les rectes $\frac{x-1}{2}=y-4=z$ i $\frac{x-2}{-3}=\frac{y-1}{4}=\frac{z-2}{-1}$
  \item El pla que passa pels punts $(2,2,1)$ i $(-1,1,-1)$ i és perpendicular al pla $2x-3y-z=3$
\end{enumerate}
 
\Answer Usarem que els coeficients de l'equació cartesiana del pla $Ax+By+Cz=D$ corresponen amb les coordenades del seu vector normal $\uvec{n}=A\uvec{i}+B\uvec{j}+C\uvec{k}$

\begin{enumerate}
  \item El pla que passa pels punts $(0,0,0)$, $(1,2,3)$ i $(-2,3,3)$.

El vector perpendicular al pla es por calcular amb el producte vectorial de dos vectors del pla com, per exemple: $\uvec{u}=(1,2,3)-(0,0,0)$ i $\uvec{v}=(-2,3,3)-(0,0,0)$. Així,

\[
\uvec{u} \times \uvec{v}=
\begin{vmatrix}
\uvec{i} & \uvec{j} & \uvec{k} \\
1 & 2 & 3\\
-2 & 3 & 3
\end{vmatrix}=
(6\uvec{i}-6\uvec{j}+3\uvec{k})-(9\uvec{i}+3\uvec{j}-4\uvec{k})=-3\uvec{i}-9\uvec{j}+7\uvec{k}
\]

Per tant, l'equació serà de la forma: $-3x-9y+7z=D$. Com que el pla passa pel punt $(0,0,0)$, l'equació buscada és:
\[
-3x-9y+7z=0
\]

  \item El pla que passa pel punt $(2,1,2)$ i té per vector normal $\uvec{n}=2\uvec{i}+3\uvec{j}-\uvec{k}$.

El pla tindrà la forma $2x+3y-z=D$. Substituint el punt donat obtenim el valor de $D$:
\[
2\cdot2+3\cdot1-2=D=5
\]
Per tant, el pla buscat és:
\[
2x+3y-z=5
\]

  \item El pla que passa pel punt $(3,2,2)$ i és perpendicular a la recta $\frac{x-1}{4}=y+2=\frac{z+3}{-3}$

El vector normal al pla és el director de la recta i, per tant: $4x+y-3z=D$. Substituint el punt donat obtenim el valor de $D$:
\[
4\cdot3+2-3\cdot2=D=8
\]
Per tant, el pla buscat és:
\[
4x+y-3z=8
\]

  \item El pla que conté les rectes $\frac{x-1}{2}=y-4=z$ i $\frac{x-2}{-3}=\frac{y-1}{4}=\frac{z-2}{-1}$

Usarem els  vectors directors de les dues rectes per trobar el vector normal al pla.

\[
\uvec{u} \times \uvec{v}=
\begin{vmatrix}
\uvec{i} & \uvec{j} & \uvec{k} \\
2 & 1 & 1\\
-3 & 4 & -1
\end{vmatrix}=
(-1\uvec{i}-3\uvec{j}+8\uvec{k})-(4\uvec{i}-2\uvec{j}-3\uvec{k})=-5\uvec{i}-\uvec{j}+11\uvec{k}
\]

El pla tindrà la forma $-5x-y+11z=D$. Podem ara agafar qualsevol punt dels que estan continguts a les dues rectes. Per exemple, agafant-lo de l'equació contínua de la primera:
\[
-5\cdot1-4+11\cdot0=D=-9
\]
Per tant, el pla buscat és:
\[
-5x-y+11z=-9
\]



  \item El pla que passa pels punts $(2,2,1)$ i $(-1,1,-1)$ i és perpendicular al pla $2x-3y-z=3$

Un dels vecotrs directors del pla buscat serà el normal del pla perpendicular: $\uvec{u}=(2,-3.-1)$. L'altre, elpodem obtenir a partir dels dos punts donats: $\uvec{v}=(-1,1,-1)-(2,2,1)=(-3,-1,-2)$. Ara podem calcular el vector normal al pla demanat:

\[
\uvec{u} \times \uvec{v}=
\begin{vmatrix}
\uvec{i} & \uvec{j} & \uvec{k} \\
2 & -3 & -1\\
-3 & -1 & -2
\end{vmatrix}=
(6\uvec{i}+3\uvec{j}-2\uvec{k})-(\uvec{i}-4\uvec{j}+9\uvec{k})=5\uvec{i}+7\uvec{j}-11\uvec{k}
\]

El pla tindrà la forma $5x+7y-11z=D$. Podem ara agafar qualsevol punt dels que estan continguts a les dues rectes.
\[
5\cdot(-1)+7\cdot1-11\cdot(-1)=D=13
\]
Per tant, el pla buscat és:
\[
5x+7y-11z=13
\]


\end{enumerate}
