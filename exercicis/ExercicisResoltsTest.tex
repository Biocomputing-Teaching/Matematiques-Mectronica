\documentclass[12pt]{article}
\input{../common/common.tex}
\usepackage{amsmath, amssymb, amsthm}

\usepackage[catalan]{babel}
\usepackage{graphicx}
\graphicspath{{../figures/}}

\begin{document}

\lstset{language=Matlab,%
    basicstyle=\color{red},
    breaklines=true,%
    morekeywords={matlab2tikz},
    keywordstyle=\color{blue},%
    morekeywords=[2]{1}, keywordstyle=[2]{\color{black}},
    identifierstyle=\color{black},%
    stringstyle=\color{purple},
    commentstyle=\color{green},%
    showstringspaces=false,%without this there will be a symbol in the places where there is a space
    numbers=left,%
    numberstyle={\tiny \color{black}},% size of the numbers
    numbersep=9pt, % this defines how far the numbers are from the text
    emph=[1]{for,end,break},emphstyle=[1]\color{red}, %some words to emphasise
    emph=[2]{word1,word2}, emphstyle=[2]{style},
}


\title{Exercicis Resolts \\ \large MATEMÀTIQUES I \\ Grau en Enginyeria Mecatrònica 
\author{Jordi Villà i Freixa}\thanks{Adreça electrònica: \texttt{jordi.villa@uvic.cat}}
\begin{center}\includegraphics[width = 60mm]{FCTE}\end{center}}
\date{Darrera modificació: \today}
\maketitle

\tableofcontents
\newpage

document per testejar nous exercicis abans de ficar-los al ExercicisResolts.tex
%
\begin{ExerciseList}

\section{Derivades}

\Exercise Troba els punts estacionaris, i el seu tipus, de la funció $f(x)=2x^3-3x^2-36x+2$
\label{ex:puntsestacionaris1}
\Answer


En trobem primer la derivada i la igualem a zero per tal de trobar els punts estacionaris:

\[f'(x)=6x^2-6x-36=6(x^2-x-6)=0\]

\[x=\frac{1\pm\sqrt{1^2-4\cdot1\cdot(-6)}}{2}=\left\{\begin{array}{c}3\\-2\end{array}\right.\]

Un cop trobats, per saber de quin tipus són fem la segona derivada i els hi substituïm:
\[f''(x)=6(2x-1)\]
\[f''(x=3)=30>0\]
\[f''(x=-2)=-30<0\]

Per tant, la funció té un màxim a $x=-2$ i un mínim a $x=3$, com es pot veure al gràfic de la Figura \ref{fig:puntsestacionaris1}.

\begin{center}
\begin{minipage}{8cm}
\begin{tikzpicture}[>=latex]
  \begin{axis}[%axis equal,
    axis lines=middle, 
    xmin=-5, xmax=5,
    ymin=-100, ymax=50,
    xtick={-5,...,5},
    ytick={-100,-50,0,50},
    grid=major,
    samples=222,
    xlabel={$x$},
    xlabel style = {anchor=south east},
    ylabel={$y$},
    ylabel style = {anchor=west},
    enlarge x limits={abs=0.5},
  ]
  \addplot[draw=green, thick, mark=none,domain=-5:5]{2*x^3-3*x^2-36*x+2}; 
  \addplot[draw=blue, thick, mark=none,domain=-5:5]{6*x^2-6*x-36};    
  \addplot[draw=red, thick, mark=none,domain=-5:5]{12*x-6};

  \end{axis}
  \end{tikzpicture}
  \captionof{figure}{$f(x)$ (verd), $f'(x)$ (blau) i $f''(x)$ (vermell) de l'Exercici \ref{ex:puntsestacionaris1}.}
  \label{fig:puntsestacionaris1}
\end{minipage} 
\end{center}

Aquest codi permet resoldre l'exercici a \texttt{MATLAB}:
\begin{lstlisting}[style=Matlab-editor]
  % resolució exercici
  syms x
  f=2*x^3-3*x^2-36*x+2
  fplot(f)
  f1=diff(f)
  f2=diff(f1)
  solve(f1==0)
  \end{lstlisting}


\blacksquare 
\Exercise Troba la derivada $\frac{dy}{dx}$ si $x^3-3xy+y^3=2$
\label{ex:derivada1}
\Answer


Es tracta d'una funció implícita i farem la derivada com a tal:

\begin{eqnarray*}
  \frac{d}{dx}(x^3-3xy+y^3)&=&\frac{d}{dx}(2)\\
  3x^2-(3y+3x\frac{dy}{dx})+3y^2\frac{dy}{dx}&=&0\\
  (3x^2-3y)+(3y^2-3x)\frac{dy}{dx}&=&0\\
  \frac{dy}{dx}&=&\frac{-3x^2+3y}{3y^2-3x}=\frac{y-x^2}{y^2-x}
\end{eqnarray*}

Aquest codi permet resoldre l'exercici a \texttt{MATLAB}:
\begin{lstlisting}[style=Matlab-editor]
  % resolució exercici
  syms y(x) DY
  eqn=x^3-3*x*y+y^3==0
  dy=diff(y)
  deqn=diff(eqn,x)
  Deqn = subs(deqn, dy, DY);
  DYsol = simplify( solve(Deqn, DY) );
  disp(DY == DYsol)
  \end{lstlisting}

\blacksquare 
\Exercise Troba la derivada $\frac{dy}{dx}$ si $y=\sin(3x+4y)$
\label{ex:derivada1}
\Answer


Es tracta d'una funció implícita i farem la derivada com a tal:

\begin{eqnarray*}
  \frac{d}{dx}y&=&\frac{d}{dx}\sin(3x+4y)\\
  \frac{dy}{dx}&=&\cos(3x+y)\left(3+4\frac{dy}{dx}\right)\\
  \frac{dy}{dx}\left(1-4\cos(3x+y)\right)&=&3\cos(3x+y)\\
  \frac{dy}{dx}&=&\frac{3\cos(3x+y)}{1-4\cos(3x+y)}
\end{eqnarray*}

Aquest codi permet resoldre l'exercici a \texttt{MATLAB}:
\begin{lstlisting}[style=Matlab-editor]
  % resolució exercici
  syms y(x) DY
  eqn=y==sin(3*x+4*y)
  dy=diff(y)
  deqn=diff(eqn,x)
  Deqn = subs(deqn, dy, DY)
  DYsol = simplify( solve(Deqn, DY) )
  disp(DY == DYsol)
  \end{lstlisting}

\blacksquare 
\Exercise Troba l'error comès per aproximar el valor de $e^0.1$ a partir de l'aproximació lineal de la funció a l'origen. 
\label{ex:taylor1}
\Answer

Cal recordar que, en tant que infinitèssims equivalents,  $e^x \sim x+1$ per a valors de $x\rightarrow0$ (veure Figura \ref{fig:taylor1}).

\begin{center}
  \begin{minipage}{9cm}
  \begin{tikzpicture}[>=latex]
    \begin{axis}[%axis equal,
      axis lines=middle, 
      %xmin=-5, xmax=5,
      %ymin=-100, ymax=50,
      %xtick={-5,...,5},
      %ytick={-100,-50,0,50},
      grid=major,
      samples=222,
      xlabel={$x$},
      xlabel style = {anchor=south east},
      ylabel={$y$},
      ylabel style = {anchor=west},
      enlarge x limits={abs=0.5},
    ]
    \addplot[draw=green, thick, mark=none,domain=-2:2]{exp(x)}; 
    \addplot[draw=red, thick, mark=none,domain=-2:2]{x+1};
  
    \end{axis}
    \end{tikzpicture}
    \captionof{figure}{Les gràfiques de les funcions $y=e^x$ (verda) i $y=x+1$ (vermella) coincideixen en el seu valor i en la seva primera derivada al voltant de $x=0$ (són infinitèssims equivalents).}
    \label{fig:taylor1}
  \end{minipage} 
  \end{center}

  Per tant, per a valors propers a zero, com és el cas de $x=0.1$, teniom que
  \[e^{0.1} \sim 0.1 +1 = 1.1\]

  Podem comprovar amb la nostra calculadora que el valor correcte fins a 4 xifres decimals és $e^{0.1}=1.1052$.


\blacksquare 
\Exercise Com varia la temperatura d'un recinte, que ve donada per la funció $T=(2xy+2z^2)^{\circ}C$ en fer un desplaçament a partir del punt $P=(1,5,1)$ d'una unitat de longitud en la direcció del vector $\uvec{a}=2\uvec{i}+\uvec{j}$
\label{ex:derivadadireccional1}
\Answer

Notem primer que la funció depèn de tres coordenades, $T(x,y,z)$. 
Ens demanen avaluar la derivada direccional de la funció en la direcció del vector $\vec{a}=2\uvec{i}+\uvec{j}$. Hem de fer, doncs, dues coses:
\begin{itemize}
  \item Trobar el gradient de la funció:
  \[
    \grad{T(x,y,z)}=\begin{pmatrix}\frac{dT}{dx}\\\frac{dT}{dy}\\\frac{dT}{dz}\end{pmatrix}= \begin{pmatrix}2y\\2x\\4z\end{pmatrix} 
  \]
  En el punt $P=(1,5,1)$ tenim que $\grad{T(1,5,1)}=(10,2,4)$.
  \item Trobar el vector unitari en la direcció del vector donat $\vec{a}=2\uvec{i}+\uvec{j}$:
  \[
    \hat{\uvec{a}}=  \frac{\uvec{a}}{\norm{\uvec{a}}}=\frac{1}{\sqrt{5}}(2,1,0)
  \]
\end{itemize}

així, la derivada direccional serà el producte escalar:
\[
  D_{\uvec{a}}(P)=  \grad{T(x,y,z)} \cdot \hat{\uvec{a}} = (10,2,4) \cdot \frac{1}{\sqrt{5}}(2,1,0) = 22\frac{1}{\sqrt{5}}
\]

Aquest codi permet resoldre l'exercici a \texttt{MATLAB}:
\begin{lstlisting}[style=Matlab-editor]
  % resolució exercici
  syms y(x) DY
  eqn=y==sin(3*x+4*y)
  dy=diff(y)
  deqn=diff(eqn,x)
  Deqn = subs(deqn, dy, DY)
  DYsol = simplify( solve(Deqn, DY) )
  disp(DY == DYsol)
  \end{lstlisting}

\blacksquare 

\end{ExerciseList}

\end{document}
