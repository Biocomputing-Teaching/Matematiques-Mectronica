\documentclass[]{article}

%These tell TeX which packages to use.

\input{../common.tex}

%\cfoot{\bf Pràctica 1 MATLAB}

\title{Pràctica MATLAB 1 \\ Matemàtiques I \\ Grau en Enginyeria Mecatrònica \\\vspace{2cm} \includegraphics[width=7cm]{FCTE}}
\author{Jordi Villà i Freixa}
\date{26 d'Octubre de 2022}

\begin{document}

\maketitle

Exercicis a resoldre usant MATLAB:
\begin{itemize}
  \item Cal penjar al moodle un fitxer MATLAB *.mlx que contingui totes les operacions i les corresponents explicacions
  \item Es treballa per parelles
  \item Cada exercici ha de ser executable i donar el resultat correcte per poder ser valorat positivament. Qualsevol exercici que no es pugui executar bé tindrà valoració zero.
  \item Enumereu clarament al fitxer *.mlx cada exercici usant les eines d'edició d'aquest tipus de fitxers.\footnote{Podeu trobar exemples de fitxer *.mlx a \href{https://drive.matlab.com/sharing/8337394b-f49e-4b91-b153-51e8135aa605}{la carpeta compartida a MATLAB Drive}}
  \item Totes les qüestions referents a aquests exercicis s'han d'adreçar al fòrum de l'assignatura.
  \item Data límit lliurament: 30 de Novembre a les 23:59
  \item Cada exercici val 1 punt.
\end{itemize}

\begin{enumerate}
% 2021 \item Estudia el domini, la continuïtat i la derivabilitat de la funció $f_1(x)=\arccos \left( \frac{x^2-2}{3x^2+2x+1}\right)$
  \item Estudia el domini, la continuïtat i la derivabilitat de la funció $f(x)=\frac{\sin{x}+\cos{x}}{\sin{x}-\cos{x}}$
  \item Calcula aquests límits (si no existeixen, mostra perquè no):
   \[
    \begin{array}{|l|l|}
      % \lim_{x \rightarrow 2} \frac{(x^2+4)\sin{x}}{x-2} &
      % \lim_{x \rightarrow 2} \frac{(x^2+4)}{x-2}\\
      % \lim_{x \rightarrow 2} \frac{(x^2-4)\sin{x}}{x-2} &
      % \lim_{x \rightarrow 2} \frac{\sin{x}}{x-2}\\
      % \lim_{x \rightarrow 2} (x^2+4)\sin{x} &
      % \lim_{x \rightarrow 2} \sin{x}
      \hline
      \lim_{x \rightarrow 2} (x^2+4)\sin{x} &
      \lim_{(x,y) \rightarrow (0,0)} \frac{x^2y}{x^2+y^2}\\
      \lim_{(x,y) \rightarrow (0,0)} f(x,y) \, \mathrm{si}\, f(x,y)=\begin{cases}\frac{2x-y^2}{2x^2+y} & y \neq -2x^2\\0& y = -2x^2\end{cases} &
      \lim_{x \rightarrow 2} \frac{\sin{x}}{x-2}\\\hline
    \end{array}
    \]
  % 2021 \item Trobar els punts màxims i minims (si existeixen), de la funció $f(x)=(2x^3)e^{-x^4}$, així com les regions de corbatura positiva (còncaves) o negativa (convexes)
  \item Trobar els punts màxims i minims (si existeixen), de la funció $f(x)=\left((x-1)(x-2)^2\right)^{1/3}$, així com les regions de corbatura positiva (còncaves) o negativa (convexes)
  \item Calcula les integrals (si es pot)
  \begin{itemize}
    % \item $\int_{-\pi}^{6\pi} \frac{\cos^2{x}}{1+\sin^2{x}}$
    % \item $\int_{-\infty}^{\infty} x e^{-x^2}$
    % \item $\int_0^{\infty} \frac{1}{x^2}$
    \item $\int_{-1}^{1} (2x^2-x^3) dx$
    \item $\int_{0}^{\infty} e^{-x}dx$
    \item $\int_1^{\infty} x \sin{x}dx$

  \end{itemize}
  % 2021 \item Calcula l'àrea que tanquen les funcions $y=x$, $y=\ln{x}$ i l'eix de les ordenades
  \item Calcula l'àrea que tanquen les corbes $y=10$, $x=1$ i $y=e^x$ i l'eix de les ordenades
  \item Estudia la continuïtat i derivabilitat de la funció $f(x,y)=\frac{3y^2x}{5x^2-10y^4}$, així com el gradient al punt $(1,1)$ i la derivada direccional en aquest punt en la direcció del vector $(-1,1)$. Estudia els punts crítics de la funció (si existeixen).
\end{enumerate}

\end{document}
